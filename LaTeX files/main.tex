\documentclass[conference,a4paper]{IEEEtran}
\usepackage[left=1.43cm,right=1.43cm,top=1.8cm,bottom=4.21cm]{geometry}
\IEEEoverridecommandlockouts

\usepackage[english]{babel}
\usepackage{amsmath, amssymb}
\usepackage{graphicx}
\usepackage{subfiles}
\usepackage{soul}
\usepackage{subcaption}
\usepackage{epstopdf}
\usepackage{xcolor}
\usepackage{multirow}
\usepackage{booktabs}
\usepackage{float}
\usepackage{algorithmic}
\DeclareMathOperator*{\argmin}{argmin}
\DeclareMathOperator*{\argmax}{argmax}
\algsetup{linenosize=\small}
\usepackage[linesnumbered,ruled]{algorithm2e}

\def\BibTeX{{\rm B\kern-.05em{\sc i\kern-.025em b}\kern-.08em
    T\kern-.1667em\lower.7ex\hbox{E}\kern-.125emX}}

\usepackage{verbatim}

\begin{document}

\title{Coordinated Multi-Armed Bandits for Improved Spatial Reuse in Wi-Fi}

\author{
\IEEEauthorblockN{Francesc Wilhelmi$^{\star\dagger}$, Boris Bellalta$^{\dagger}$, Szymon Szott$^{\mathsection}$, Katarzyna Kosek-Szott$^{\mathsection}$, Sergio Barrachina-Mu\~noz$^{\ddagger}$\vspace{0.1cm}
}
%
\IEEEauthorblockA{$^{\star}$\emph{Nokia Bell Labs, Stuttgart, Germany}}
\IEEEauthorblockA{$^{\dagger}$\emph{Universitat Pompeu Fabra, Barcelona, Spain}}
\IEEEauthorblockA{$^{\mathsection}$\emph{AGH University of Krakow, Poland}}
\IEEEauthorblockA{$^{\ddagger}$\emph{Centre Tecnològic de Telecomunicacions de Catalunya, Barcelona, Spain}}
%
\IEEEauthorblockN{\thanks{This paper is supported by the CHIST-ERA Wireless AI 2022 call MLDR project (ANR-23-CHR4-0005), partially funded by AEI and NCN under projects PCI2023-145958-2 and DEC-2023/05/Y/ST7/00004, respectively. B. Bellalta's contribution is supported by Wi-XR PID2021123995NB-I00 (MCIU/AEI/FEDER,UE) and MdMCEX2021-001195-M/ AEI /10.13039/501100011033. S. Barrachina-Muñoz is supported by Grant PID2021-126431OB-I00 funded by MCIN/AEI/ 10.13039/501100011033 (ERDF A way of making Europe) and by 6GE2E (2021 SGR 00770) from Generalitat de Catalunya.}}
}



\bstctlcite{IEEEexample:BSTcontrol}

\maketitle

\begin{abstract}
Multi-Access Point Coordination (MAPC) and Artificial Intelligence and Machine Learning (AI/ML) are expected to be key features in future Wi-Fi, such as the forthcoming IEEE 802.11bn (Wi-Fi~8) and beyond. In this paper, we explore a coordinated solution based on online learning to drive the optimization of Spatial Reuse (SR), a method that allows multiple devices to perform simultaneous transmissions by controlling interference through Packet Detect (PD) adjustment and transmit power control. In particular, we focus on a Multi-Agent Multi-Armed Bandit (MA-MAB) setting, where multiple decision-making agents concurrently configure SR parameters from coexisting networks by leveraging the MAPC framework, and study various algorithms and reward-sharing mechanisms. We evaluate different MA-MAB implementations using Komondor, a well-adopted Wi-Fi simulator, and demonstrate that AI-native SR enabled by coordinated MABs can improve the network performance over current Wi-Fi operation: mean throughput increases by 15\%, fairness is improved by increasing the minimum throughput across the network by 210\%, while the maximum access delay is kept below 3~ms.
\end{abstract}


\begin{IEEEkeywords}
Artificial Intelligence, IEEE 802.11, Machine Learning, Multi Access Point Coordination, Multi-Armed Bandits, Spatial Reuse, Wi-Fi
\end{IEEEkeywords}

\subfile{sections/1_introduction}

\subfile{sections/2_related_work}

\subfile{sections/3_mapc_solution}

\subfile{sections/4_performance_evaluation}

\subfile{sections/5_conclusions}

\bibliographystyle{IEEEtran}
\bibliography{bib}

\end{document}