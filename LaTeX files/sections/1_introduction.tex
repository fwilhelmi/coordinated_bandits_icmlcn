\section{Introduction}
\label{sec:introduction}

The ever increasing demand for high performance and reliability in wireless networks has driven the development of sophisticated features for technologies like IEEE 802.11 (commercially known as Wi-Fi). This includes the ongoing work on Multi-Access Point Coordination (MAPC) for Wi-Fi~8~\cite{reshef2022future} by the IEEE 802.11bn task group, which addresses issues associated with the coexistence of multiple APs, where achieving optimal performance remains a significant challenge. MAPC entails a paradigm shift with respect to current distributed channel operations in Wi-Fi, since it is foreseen as a mechanism for enabling APs to collaborate and optimize resources across the network by exchanging information. Some potential features enabled by MAPC are Coordinated Spatial Reuse (C-SR), Coordinated Beamforming (C-BF), and Coordinated Orthogonal Frequency-Division Multiple Access (C-OFDMA)~\cite{verma2023survey}.

Regarding C-SR~\cite{wilhelmi2023throughput}, it aims to evolve the current Overlapping Basic Service Set Packet Detect (OBSS/PD) SR mechanism, which was introduced in IEEE 802.11ax (2020)~\cite{wilhelmi2021spatial}. OBSS/PD SR builds on top of BSS coloring (allowing for fast inter-BSS transmission identification) to unlock the usage of OBSS/PD thresholds that are less restrictive than Clear Channel Assessment (CCA) so that more Transmission Opportunities (TXOPs) can be created. However, OBSS/PD SR has seen little adoption in commercial equipment so far because of the moderate performance improvements compared to Distributed Coordination Function (DCF). One of the main limitations of OBSS/PD SR operation stems from its decentralized implementation, which leads to imposing too restrictive constraints in terms of transmit power, thus significantly limiting the mechanism's achievable gains. C-SR aims at overcoming OBSS/PD SR decentralization by leveraging MAPC inter-BSS communication (e.g., to exchange interference measurements), thus allowing multiple BSSs to perform efficient simultaneous transmissions. Significant research and standardization efforts are already being put into C-SR, which recently became a candidate feature for IEEE 802.11bn (Wi-Fi 8) after being accepted into the group's Specification Framework Document (SFD). However, there is still no consensus on how C-SR will materialize, so its implementation remains open.

This paper investigates the potential of MAPC-enabled coordinated Multi-Armed Bandits (MABs) for enhancing SR in coordinated Wi-Fi networks. Our approach presents a promising alternative to C-SR, with the aim of achieving similar network-wide benefits while reducing the complexity of joint Transmit Power Control (TPC) and PD adjustment decisions. MAB is a popular framework for sequential decision-making under uncertainty, which makes it well-suited for overcoming the complex OBSS interactions that occur in Wireless Local Area Networks (WLANs). This underlying complexity is precisely the main motivation for adopting Artificial Intelligence~(AI) and Machine Learning~(ML) techniques such as MABs, which are expected to address the dynamic and unpredictable nature of wireless environments, for which static approaches (even if coordinated) can fail. In fact, the adoption of AI/ML in the IEEE 802.11 is gaining momentum with the establishment of the AI/ML Study Group~(SG) and the AI/ML Standing Committee~(SC)~\cite{wilhelmi2024machine}. This paper's contributions are as follows:
\begin{itemize}
    \item We propose a coordinated MAB solution to learn the best SR policies (combining PD and transmit power configurations) on a scenario basis, so that multiple neighboring BSSs can wisely identify favorable TXOPs. Our coordinated MAB solution leverages the MAPC framework to enable inter-AP MAB communication.
    %
    \item We propose and study a set of coordinated MAB algorithms and policies based on various reward-sharing strategies for improving SR. 
    %
    \item We conduct simulations to study the performance of these algorithms as well as delve into their fairness. We also compare the coordinated MAB solutions to baseline approaches, including the OBSS/PD SR operation and non-coordinated bandits.
    %
    \item We showcase the effectiveness of coordinated MAB algorithms for improving overall network performance in various Wi-Fi network scenarios and provide guidelines on how and when to use them effectively. 
\end{itemize}