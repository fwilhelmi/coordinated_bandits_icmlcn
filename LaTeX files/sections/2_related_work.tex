\section{Related Work}
\label{sec:related_work}

The achievable performance of C-SR mechanisms has been evaluated in multiple ways and through various implementations. In~\cite{nunez2024spatial}, for example, a model-based evaluation of \mbox{C-SR} was shown to double DCF performance in some scenarios. Similarly, the work in~\cite{nunez2023multi} delved into different scheduling strategies for performing C-SR once MAPC groups are established. In that regard, the authors showed that AP-centric strategies (e.g., aiming to prioritize highly loaded APs) perform better than group-centric ones (e.g., aiming to maximize aggregate performance). Another implementation of C-SR in ns-3 demonstrated the superiority of C-SR against uncoordinated OBSS/PD SR, with a throughput improvement factor of $2.3$~\cite{imputato2024beyond}. Furthermore, a prototype solution of a centralized C-SR mechanism was presented in~\cite{haxhibeqiri2024coordinated, haxhibeqiri2024commag} to bring $33$\% goodput enhancements when enabled. Other studies confirm the appeal of applying C-SR when combined with other techniques. An example is the work in \cite{nunez2022txop}, which showed the benefits of combining C-SR with C-TDMA. In particular, C-SR with C-TDMA was shown to improve throughput by 30\% and reduce latency by a factor of 2, compared to applying C-SR only. 

Another prominent research line, as part of future AI-native radios~\cite{wilhelmi2024machine}, is AI-driven SR, which aims to overcome the limitations of the static approaches currently proposed for C-SR. In particular, existing proposals use radio measurements such as the Received Signal Strength Indicator (RSSI) statically (e.g., averaged values) to derive C-SR policies that allow for an acceptable Signal-to-Noise Ratio (SNR). However, such an approach may lose effectiveness in real deployments where the variability of both RSSI and SNR is very high. Some works exploring AI-driven SR solutions are \cite{wilhelmi2019potential, wilhelmi2019collaborative}, where different MAB solutions were proposed and studied to drive the optimization of SR parameters such as PD and transmit power. The exploration-exploitation paradigm embedded in MAB showed potential for addressing the non-stationarity experienced in OBSSs implementing SR solutions. However, these solutions were based on decentralized mechanisms (no communication between agents was provided), which led to several issues as a result of the competition arising among networks. MAB solutions have also been assessed in other Wi-Fi problems, such as channel bonding, demonstrating their effectiveness as a lightweight, efficient, ready-to-use solution~\cite{barrachina2021multi}. In this work, similar to what was done in~\cite{wilhelmi2019potential, wilhelmi2019collaborative}, we focus on an AI-native solution to improve SR. As a step forward, we study the advantages that coordination may bring to the operation of decentralized agents, which are assumed to cooperate within an MAPC framework.